% Manuscript: submission to Cognitive Science
% jefan@stanford.edu, yamins@mit.edu, nicholas.turk-browne@yale.edu

\documentclass[11pt,letterpaper]{article}

% \usepackage{cogsci}
\usepackage{pslatex}
\usepackage{apacite}
\usepackage{graphicx}
\usepackage{textcomp}
\usepackage[all]{nowidow}
\usepackage[font=small,skip=0pt]{caption}
\usepackage{mathtools}
\usepackage{float}
\usepackage{courier}
\usepackage{placeins}
\usepackage{sidecap}
\usepackage{lineno}
\usepackage{authblk}
\usepackage{caption,setspace}

\usepackage{color}
\definecolor{Blue}{RGB}{50,50,200}
\newcommand{\revised}[1]{\textcolor{Blue}{#1}}

\captionsetup{font={normal,stretch=1.1}}
\setlength{\parindent}{2em}
\setlength{\parskip}{0.5em}
\renewcommand{\baselinestretch}{1.5}
\usepackage[letterpaper, margin=0.9in]{geometry}

% \setlength{\intextsep}{9pt}
\setlength{\belowcaptionskip}{-\baselineskip}\addtolength{\belowcaptionskip}{1.6ex}

\title{Consequences of visual production training for neural object representations}
\date{}

\author[1,3]{Judith E. Fan}
\author[2]{Jeffrey Wammes}
\author[3]{Jordan B. Gunn}
\author[3]{Rachel S. Lee}
\author[1]{Daniel L. K. Yamins}
\author[3]{Kenneth A. Norman}
\author[2]{Nicholas B. Turk-Browne}

\affil[1]{Department of Psychology, Stanford University, Stanford, CA 94305}
\affil[2]{Department of Psychology, Yale University, New Haven, CT 06520}
\affil[3]{Department of Psychology, Princeton University, Princeton, NJ 08544}

\renewcommand\Authands{ and }

\begin{document}
\maketitle

\begin{abstract}
\hyphenpenalty=1000 {\revised{Intro sentences go here.}The current study directly evaluates how practice drawing objects affects their underlying representation in ventral temporal cortex using fMRI. All three phases (pre, training, post) of the study were scanned. During training, participants alternately drew two objects (e.g., table, bed) on an MR-safe tablet. Before and after training, they viewed these and two other control objects (e.g., chair, bench), so that we could obtain estimates of the neural representation of each object. \revised{More methods and results sentences go here.}}


\end{abstract}
\textbf{Keywords:}
communication; drawing; learning; perception and action; objects

\newpage
\linenumbers

\section*{Introduction}

\section*{Methods}

\subsection*{Stimuli}

\subsection*{Task and procedure}

\subsubsection*{Recognition task}

\subsubsection*{Production task}

\subsection*{fMRI data acquisition and preprocessing}

\subsection*{Measuring object representation during recognition and production}

\subsection*{Connectivity pattern similarity analysis}

\subsection*{Searchlight analysis}

\section*{Results}

\subsection*{Shared representations during recognition and production}

\subsection*{Sustained selection of target object during drawing within early visual, parietal, frontal regions}

\subsection*{Sustained selection of target object during drawing between regions}

\subsection*{Relationship between target selection and representational differentiation}

\subsection*{Distinct dynamics in target object representation in visual and frontal regions during drawing}

% 1. generalizing object classifier from recog to drawing
%     1. reliable in V1, V2, and LOC (weakly in fusiform)
% 2. sustained preference for target vs. foil
%     1. consistently over time in EVC: V1, V2 especially
%     2. drawing behavior: using VGG to separately quantify target evidence, in terms of sketch recognizability
% 3. relationship between target vs. foil preference and prepost differentiation in these regions
%     1. temporal development of this relationship (TBD)
%     2. for regions showing negative relationship (e.g., fusiform), does this reflect high vs. low target selectivity?
%     3. pre-post differentiation: clusters of regions that show similar pre-post differentiation behavior within individual
%     4. relationship between foil inhibition in hippocampus and pre-post differentiation in V1, V2
% 4. object-specific information in dorsal stream regions also
%     1. but different dynamics in sensory vs. anterior regions
%     2. connectivity profile between early visual cortex, parietal, and frontal regions contains information distinguishing target from foil
%     3. target evidence during drawing: some clustering-based evidence for 2-component visual production system (TBD)

\section*{Discussion}

\section*{Code availability} The code for the analyses presented in this article will be made publicly available in a Github repository upon acceptance of this manuscript.

\section*{Data availability} The data presented in this article will be made publicly available in a figshare repository upon acceptance of this manuscript.

\section*{Acknowledgements}

This work was supported by NSF GRFP DGE-0646086, NIH R01 EY021755, R01 MH069456, and the David A. Gardner '69 Magic Project at Princeton University. Thanks to the Computational Memory and Turk-Browne labs for helpful comments.

\section*{Author contributions statement}

J.E.F., D.L.K.Y., N.B.T.-B., K.A.N. designed the study. J.E.F. performed the experiments. J.E.F., J.D.W., J.B.G., R.S.L. conducted analyses. J.E.F., J.D.W., J.B.G., R.S.L., K.A.N., and N.B.T.-B. planned analyses, interpreted results, and wrote the paper.

\section*{Additional information}

The authors declare no competing interests.

\bibliographystyle{apacite}

\setlength{\bibleftmargin}{.125in}
\setlength{\bibindent}{-\bibleftmargin}

\newpage
\bibliography{references}

\end{document}
